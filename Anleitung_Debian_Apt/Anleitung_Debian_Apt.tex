\documentclass[parskip=full-]{scrartcl}
\usepackage[utf8]{inputenc}
\usepackage[ngerman]{babel}
\usepackage{multicol}
\usepackage{amsmath}
\usepackage{amsfonts}
\usepackage{amssymb}
\usepackage{graphicx}
\usepackage{float}
\usepackage{enumitem} 
\usepackage{lmodern}
\usepackage{url}
\usepackage[pdftex,
    pdfauthor={Hoffmeier, Hoopmann, Johennecken},
    pdftitle={Anleitung Debian APT},
    pdfsubject={Anleitung Debian APT},
    pdfkeywords={Anleitung Debian APT},
    pdfproducer={Latex with hyperref},
pdfborder={0 0 0}]{hyperref}
\usepackage{caption}
\usepackage{subcaption}
\usepackage[skip=2pt,font=scriptsize]{caption}
\renewcommand*\familydefault{\sfdefault}


\renewcaptionname{ngerman}{\contentsname}{Inhalt}
\renewcaptionname{ngerman}{\listfigurename}{Abbildungen}
\renewcaptionname{ngerman}{\listtablename}{Tabellen}
\renewcaptionname{ngerman}{\figurename}{Abb.}
\renewcaptionname{ngerman}{\tablename}{Tab.}

\author{Hoffmeier, Hoopmann, Johennecken}
\title{Anleitung Debian APT}
\date {\today}

\usepackage{listings}
\usepackage{color}

\definecolor{mygreen}{rgb}{0,0.6,0}
\definecolor{mygray}{rgb}{0.5,0.5,0.5}
\definecolor{mymauve}{rgb}{0.58,0,0.82}

\lstset{ %
    backgroundcolor=\color{white},   % choose the background color; you must add \usepackage{color} or \usepackage{xcolor}
    basicstyle=\footnotesize,        % the size of the fonts that are used for the code
    breakatwhitespace=true,         % sets if automatic breaks should only happen at whitespace
    breaklines=true,                 % sets automatic line breaking
    captionpos=b,                    % sets the caption-position to bottom
    commentstyle=\color{mygreen},    % comment style
    deletekeywords={...},            % if you want to delete keywords from the given language
    %escapeinside={\%*}{*)},          % if you want to add LaTeX within your code
    extendedchars=true,              % lets you use non-ASCII characters; for 8-bits encodings only, does not work with UTF-8
    frame=single,                    % adds a frame around the code
    %keepspaces=true,                 % keeps spaces in text, useful for keeping indentation of code (possibly needs columns=flexible)
    keywordstyle=\color{blue},       % keyword style
    language=bash,                 % the language of the code
    otherkeywords={*,...},            % if you want to add more keywords to the set
    numbers=right,                    % where to put the line-numbers; possible values are (none, left, right)
    numbersep=5pt,                   % how far the line-numbers are from the code
    numberstyle=\tiny\color{mygray}, % the style that is used for the line-numbers
    rulecolor=\color{black},         % if not set, the frame-color may be changed on line-breaks within not-black text (e.g. comments (green here))
    showspaces=false,                % show spaces everywhere adding particular underscores; it overrides 'showstringspaces'
    showstringspaces=false,          % underline spaces within strings only
    showtabs=false,                  % show tabs within strings adding particular underscores
    stepnumber=1,                    % the step between two line-numbers. If it's 1, each line will be numbered
    stringstyle=\color{mymauve},     % string literal style
    tabsize=2,                       % sets default tabsize to 2 spaces
    title=\lstname                   % show the filename of files included with \lstinputlisting; also try caption instead of title
}


% Das Dokument beginnt hier
\begin{document}
\maketitle
\begin{figure}[H]
    \centering
    \includegraphics[scale=.5]{img/hs}
\end{figure}
\begin{abstract}
\end{abstract}
\tableofcontents
\newpage
% Erstelle eine Hauptüberschrift
\section{Einleitung}
\section{APT-Repoisitory hinzufügen}
Um das Repository von corsac (\url{http://molly.corsac.net/~corsac/debian/kernel-grsec/packages/}) in den Packetmanager zu integrieren muss die Datei \textit{/etc/apt/sources.list} verändert werden.

Hierzu wird 
\begin{lstlisting}
deb http://perso.corsac.net/~corsac/debian/kernel-grsec/packages/ jessie/
\end{lstlisting}
and die Datei angehängt.

Zur Überprüfung der Dateien müssen wir den PGP Schlüssel von der Seite herunterladen und mit APT bekannt machen.
\begin{lstlisting}
wget -O - http://www.corsac.net/71ef0ba8.asc | apt-key -
\end{lstlisting}

Ein Update des APT-Baumes ermöglicht uns die Installation.
\begin{lstlisting}
apt-get update
\end{lstlisting}

Nun können wir grsec und pax installieren mit:
\begin{lstlisting}
apt-get install linux-image-3.14.50-grsec+ paxctl paxtest
\end{lstlisting}

Der neue Kernel wird gleich in den Bootmanager grub eingetragen. Beim Neustart ist darauf zu achten, dass der grsec-Kernel nicht der Standartkernel von grub ist. Da seine Versionsnummer niedriger als die des normalen Debian Kernels ist, muss dieser über die erweiterten Optionen gestartet werden.

Mit dem Befehl
\begin{lstlisting}
cat /boot/config-3.14.50-grsec+ | grep -E '(PAX|GRKERN)'
\end{lstlisting}
können wir uns ansehen mit welchen Parametern der Kernel kompiliert wurde.

Mit den Befehlen
\begin{lstlisting}
paxtest kiddie
paxtest blackhat
\end{lstlisting}
können wir die funktionalität von PAX überprüfen.
\end{document}
