\section{Einleitung}
\label{sec:Einleitung}
Zweck dieser Veranstaltung ist es den Studierenden GrSecurity und PaX näher zu bringen (Seminarpräsentation vom 30.09.2015).

Zur Bearbeitung der Aufgaben wird eine virtuelle Maschine verwendet, deren Appliance kann über \fnurl{Gigamove}{https://gigamove.rz.rwth-aachen.de/d/id/xFWt3sdkPs7D2E} herunter-geladen und anschließend in VirtualBox importiert werden. Sicherheitshalber erstellen Sie vor dem ersten Verwenden der VM einen Sicherungspunkt um jederzeit den Auslieferungszustand wiederherstellen zu können.

Das Passwort für den Nutzer \textbf{john} lautet \textbf{john}.

Im Ordner \textbf{SWSec} finden Sie die benötigten Tools um die folgenden Aufgaben zu bearbeiten.

Für die Aufgaben 4-8 ist es nötig einen gepatchten Kernel zu installieren, näheres hierzu in \nameref{subsec:GrSec Kernel installieren}.

\section{Aufgaben}
\label{Aufgaben}

\subsection{Exploit verwenden}
\label{subsec:Exploit verwenden}

Im folgenden Verlauf des Praktikums müssen Sie als Nutzer \textbf{root} arbeiten, jedoch fehlt Ihnen bislang das Passwort. Ihre Aufgabe ist es mittels der Tools im Ordner \textbf{SWSec} eine Root-Shell zu erlangen.\\
Die Datei source\_me enthält beispielhaften Shellcode den Sie verwenden dürfen. ASLR ist bereits deaktiviert.
\textit{Hinweis: \textbf{exploitable} und \textbf{filer} haben das SUID-Bit gesetzt. Sie können es verwenden um mittels Puffer-Überlauf Code als Nutzer \textbf{root} auszuführen.}

Stichworte: stack buffer overflow, exploitable, getenvaddr, python2, shellcode, env-var

\subsection{Root PW ändern}
\label{subsec:Root PW aendern}

Herzlichen Glückwunsch, Sie haben vermutlich gerade eine Root-Shell erlangt und können nun das Passwort des Nutzers \textbf{root} ändern (eventuell Suchmaschinen zur Hilfe verwenden).\\
Loggen Sie sich aus und als \textbf{root} wieder ein.

\subsection{GrSec Kernel installieren}
\label{subsec:GrSec Kernel installieren}

Um das Repository von \fnurl{corsac}{http://molly.corsac.net/~corsac/debian/kernel-grsec/packages/} in den Packetmanager zu integrieren muss die Datei \textbf{/etc/apt/sources.list} verändert werden.

Dazu wird der folgende Text an die Datei an gehangen. 
\begin{lstlisting}[numbers=none, frame=single, lineskip={-2.5pt}]
deb http://perso.corsac.net/~corsac/debian/kernel-grsec/packages/ jessie/
\end{lstlisting}

Zur Verifizierung der durch das Repository bereitgestellten Pakete muss der PGP Schlüssel von der Seite heruntergeladen und APT bekanntgemacht werden.
\begin{lstlisting}[numbers=none, frame=single, lineskip={-2.5pt}]
wget -O - http://www.corsac.net/71ef0ba8.asc | apt-key -
\end{lstlisting}

Ein Update des APT-Trees ermöglicht uns die Installation.
\begin{lstlisting}[numbers=none, frame=single, lineskip={-2.5pt}]
apt-get update
\end{lstlisting}

Nun kann GrSecurity und PaX mit dem folgenden Kommando installiert werden.
\begin{lstlisting}[numbers=none, frame=single, lineskip={-2.5pt}]
apt-get install linux-image-3.14.50-grsec+ paxctl paxtest
\end{lstlisting}

Der neue Kernel wird gleich in den Bootmanager grub eingetragen. Beim Neustart ist darauf zu achten, dass der GrSec-Kernel nicht der Standardkernel von grub ist. Da seine Versionsnummer niedriger als die des normalen Debian Kernels ist, muss dieser über die erweiterten Optionen gestartet werden.

Eine Übersicht über die Parameter (nur GrSec und PaX) die beim kompilieren des Kernels benutzt wurden, lässt sich mit dem folgenden Befehl ausgeben.
\begin{lstlisting}
cat /boot/config-3.14.50-grsec+ | grep -E '(PAX|GRKERN)'
\end{lstlisting}

Mit \glqq paxtest\grqq kann das PaX-System auf Funktion überprüft werden.
\begin{lstlisting}[numbers=none, frame=single, lineskip={-2.5pt}]
paxtest kiddie
paxtest blackhat
\end{lstlisting}


\subsection{Exploit unter GrSec/PaX}
\label{subsec:Exploit unter GrSec/PaX}

Sie sollten nun den GrSec-Kernel verwenden, dies können Sie mittels \textbf{uname -a} verifizieren.

\subsubsection{Exploit PaX - 1}
\label{subsubsec:Exploit PaX - 1}
Führen Sie das Exploit wie Sie es bereits in \nameref{subsec:Exploit verwenden} getan haben erneut aus.\\
Was fällt Ihnen auf?

\subsubsection{Exploit PaX - 2}
\label{subsubsec:Exploit PaX - 2}
Verwenden Sie nun die erweiterten Dateisystemattribute um die minimal notwendigen PaX-Flags zu ermitteln die für eine Ausführung des Exploits notwendig sind.

Stichworte: setfattr, getfattr, PEMRS
\url{https://en.wikibooks.org/wiki/Grsecurity/Appendix/PaX_Flags}

\subsubsection{Bonus: Exploit RBAC - 1}
\label{subsubsec:Bonus: Exploit RBAC - 1}

Aktivieren Sie mittels \textbf{gradm -F -L /etc/grsec/learning.log} den Full-System Learning Modus von GrSec und verwenden Sie ein paar Befehle, beispielsweise verwenden Sie das Programm \textbf{exploitable} in seinem ursprünglich gedachten Kontext als Begrüßungs-Tool. Verwenden Sie es nicht um eine Shell zu erlangen.

Wenn Sie der Meinung sind ausreichend Aktionen durchgeführt zu haben (Regelreduktion GrSec, min 4x), können Sie mittels \textbf{gradm -D \&\& gradm -F -L /etc/grsec/learning.log -O /etc/grsec/policy} die Policy generieren lassen (eventuell vorher auf die admin-Rolle authentifizieren, siehe \textbf{gradm -h}).

Beheben Sie eventuell auftretende Inkonsistenzen in der Policy und aktivieren Sie anschließend das RBAC-System mittels \textbf{gradm -E} und versuchen Sie erneut das Exploit zu verwenden.

\subsubsection{Bonus: Exploit RBAC - 2}
\label{subsubsec:Bonus: Exploit RBAC - 2}

Erstellen Sie ein eigenes Subjekt für unser Programm \textbf{exploitable} und tragen Sie die benötigten Objekte und deren Modes ein um eine Ausführung des Exploits zu ermöglichen.

\url{https://en.wikibooks.org/wiki/Grsecurity/Appendix/Subject\_Modes}
\url{https://en.wikibooks.org/wiki/Grsecurity/Appendix/Object\_Modes}

\subsubsection{Bonus: Filer RBAC}
\label{subsubsec:Bonus: Filer RBAC}

Sehen Sie sich den Programmtext \textbf{filer.c} an. Auch dieser kann mittels Stack-Buffer-Overflows zur Codeausführung bewegt werden, jedoch ist dieses nicht unser Ziel.\\
Das Programm liest die Datei \textbf{/etc/logfile.txt} ein und gibt die ersten 63 Byte aus. Die Datei steht nur dem Nutzer \textbf{root} zur Verfügung, weshalb \textbf{filer} das SUID-Bit gesetzt hat.

Gibt man dem Programm einen Pfad zu einer anderen Datei als Parameter mit, so öffnet es stattdessen diese. Generieren Sie eine RBAC-Policy die es unterbindet dass die Datei \textbf{/etc/shadow} eingelesen werden darf.